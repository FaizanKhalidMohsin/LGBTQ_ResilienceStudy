\documentclass[]{article}
\usepackage{lmodern}
\usepackage{amssymb,amsmath}
\usepackage{ifxetex,ifluatex}
\usepackage{fixltx2e} % provides \textsubscript
\ifnum 0\ifxetex 1\fi\ifluatex 1\fi=0 % if pdftex
  \usepackage[T1]{fontenc}
  \usepackage[utf8]{inputenc}
\else % if luatex or xelatex
  \ifxetex
    \usepackage{mathspec}
  \else
    \usepackage{fontspec}
  \fi
  \defaultfontfeatures{Ligatures=TeX,Scale=MatchLowercase}
\fi
% use upquote if available, for straight quotes in verbatim environments
\IfFileExists{upquote.sty}{\usepackage{upquote}}{}
% use microtype if available
\IfFileExists{microtype.sty}{%
\usepackage{microtype}
\UseMicrotypeSet[protrusion]{basicmath} % disable protrusion for tt fonts
}{}
\usepackage[margin=1in]{geometry}
\usepackage{hyperref}
\hypersetup{unicode=true,
            pdftitle={Clusting Analysis Resilience Score on LGBTQ},
            pdfauthor={Faizan Khalid Mohsin},
            pdfborder={0 0 0},
            breaklinks=true}
\urlstyle{same}  % don't use monospace font for urls
\usepackage{graphicx,grffile}
\makeatletter
\def\maxwidth{\ifdim\Gin@nat@width>\linewidth\linewidth\else\Gin@nat@width\fi}
\def\maxheight{\ifdim\Gin@nat@height>\textheight\textheight\else\Gin@nat@height\fi}
\makeatother
% Scale images if necessary, so that they will not overflow the page
% margins by default, and it is still possible to overwrite the defaults
% using explicit options in \includegraphics[width, height, ...]{}
\setkeys{Gin}{width=\maxwidth,height=\maxheight,keepaspectratio}
\IfFileExists{parskip.sty}{%
\usepackage{parskip}
}{% else
\setlength{\parindent}{0pt}
\setlength{\parskip}{6pt plus 2pt minus 1pt}
}
\setlength{\emergencystretch}{3em}  % prevent overfull lines
\providecommand{\tightlist}{%
  \setlength{\itemsep}{0pt}\setlength{\parskip}{0pt}}
\setcounter{secnumdepth}{0}
% Redefines (sub)paragraphs to behave more like sections
\ifx\paragraph\undefined\else
\let\oldparagraph\paragraph
\renewcommand{\paragraph}[1]{\oldparagraph{#1}\mbox{}}
\fi
\ifx\subparagraph\undefined\else
\let\oldsubparagraph\subparagraph
\renewcommand{\subparagraph}[1]{\oldsubparagraph{#1}\mbox{}}
\fi

%%% Use protect on footnotes to avoid problems with footnotes in titles
\let\rmarkdownfootnote\footnote%
\def\footnote{\protect\rmarkdownfootnote}

%%% Change title format to be more compact
\usepackage{titling}

% Create subtitle command for use in maketitle
\providecommand{\subtitle}[1]{
  \posttitle{
    \begin{center}\large#1\end{center}
    }
}

\setlength{\droptitle}{-2em}

  \title{Clusting Analysis Resilience Score on LGBTQ}
    \pretitle{\vspace{\droptitle}\centering\huge}
  \posttitle{\par}
    \author{Faizan Khalid Mohsin}
    \preauthor{\centering\large\emph}
  \postauthor{\par}
      \predate{\centering\large\emph}
  \postdate{\par}
    \date{August 3, 2019}

\usepackage{booktabs}
\usepackage{longtable}
\usepackage{array}
\usepackage{multirow}
\usepackage{wrapfig}
\usepackage{float}
\usepackage{colortbl}
\usepackage{pdflscape}
\usepackage{tabu}
\usepackage{threeparttable}
\usepackage{threeparttablex}
\usepackage[normalem]{ulem}
\usepackage{makecell}
\usepackage{xcolor}

\begin{document}
\maketitle

\newpage

\subsection{Data}\label{data}

\begin{verbatim}
## [1] 322   3
\end{verbatim}

\begin{verbatim}
## Classes 'tbl_df', 'tbl' and 'data.frame':    322 obs. of  3 variables:
##  $ Close_Score           : num  3.83 3 2.33 4.17 3 ...
##  $ Dependent_Score       : num  3 3 2.83 3 2 ...
##  $ AttachmentAnxietyScore: num  3.5 3 2.5 4 3.67 ...
\end{verbatim}

\begin{verbatim}
##   Close_Score    Dependent_Score AttachmentAnxietyScore
##  Min.   :1.000   Min.   :1.000   Min.   :1.000         
##  1st Qu.:2.333   1st Qu.:1.833   1st Qu.:3.000         
##  Median :3.000   Median :2.500   Median :3.667         
##  Mean   :3.082   Mean   :2.541   Mean   :3.585         
##  3rd Qu.:3.833   3rd Qu.:3.167   3rd Qu.:4.500         
##  Max.   :5.000   Max.   :5.000   Max.   :5.000         
##  NA's   :1       NA's   :3
\end{verbatim}

\begin{verbatim}
## [1] 319   3
\end{verbatim}

\begin{verbatim}
## Classes 'tbl_df', 'tbl' and 'data.frame':    319 obs. of  3 variables:
##  $ Close_Score           : num  3.83 3 2.33 4.17 3 ...
##  $ Dependent_Score       : num  3 3 2.83 3 2 ...
##  $ AttachmentAnxietyScore: num  3.5 3 2.5 4 3.67 ...
\end{verbatim}

\subsection{Clusting}\label{clusting}

\includegraphics{Clustering_Analysis_in_R_files/figure-latex/clustering-1.pdf}

\includegraphics{Clustering_Analysis_in_R_files/figure-latex/unnamed-chunk-1-1.pdf}

\includegraphics{Clustering_Analysis_in_R_files/figure-latex/unnamed-chunk-2-1.pdf}

\includegraphics{Clustering_Analysis_in_R_files/figure-latex/unnamed-chunk-5-1.pdf}
\includegraphics{Clustering_Analysis_in_R_files/figure-latex/unnamed-chunk-5-2.pdf}
\includegraphics{Clustering_Analysis_in_R_files/figure-latex/unnamed-chunk-5-3.pdf}
\includegraphics{Clustering_Analysis_in_R_files/figure-latex/unnamed-chunk-5-4.pdf}

\begin{center}
Figure 6.
\end{center}

This has interesting pattern. For the characteristic: behavior levels,
the data seems to be well divided into upper and lower halves at zero,
with all the S/C being above zero and majority of C/S being below zero.

\begin{verbatim}
## List of 9
##  $ cluster     : int [1:319] 1 3 3 1 1 1 3 1 2 2 ...
##  $ centers     : num [1:3, 1:3] 3.4 1.95 3.84 2.38 1.78 ...
##   ..- attr(*, "dimnames")=List of 2
##   .. ..$ : chr [1:3] "1" "2" "3"
##   .. ..$ : chr [1:3] "Close_Score" "Dependent_Score" "AttachmentAnxietyScore"
##  $ totss       : num 939
##  $ withinss    : num [1:3] 140 92.2 148.4
##  $ tot.withinss: num 381
##  $ betweenss   : num 559
##  $ size        : int [1:3] 126 98 95
##  $ iter        : int 3
##  $ ifault      : int 0
##  - attr(*, "class")= chr "kmeans"
\end{verbatim}

\begin{verbatim}
## K-means clustering with 3 clusters of sizes 126, 98, 95
## 
## Cluster means:
##   Close_Score Dependent_Score AttachmentAnxietyScore
## 1    3.398148        2.379630               4.034392
## 2    1.945578        1.784014               4.188776
## 3    3.842105        3.536842               2.328070
## 
## Clustering vector:
##   [1] 1 3 3 1 1 1 3 1 2 2 1 1 3 3 1 3 3 1 3 3 1 3 1 2 1 3 2 1 3 2 3 1 1 3 2
##  [36] 3 1 1 1 3 1 1 2 2 1 3 1 1 3 1 2 2 2 3 2 2 3 1 2 2 2 1 3 2 1 3 3 3 2 1
##  [71] 1 1 2 2 1 1 3 3 1 3 1 2 1 2 1 3 1 3 1 1 2 2 1 2 1 2 3 1 2 2 1 1 3 3 3
## [106] 3 3 2 1 3 3 1 1 2 2 3 2 3 2 3 2 1 1 2 2 1 1 2 2 3 3 3 1 3 3 3 2 3 1 1
## [141] 2 2 1 1 2 2 2 3 2 1 1 1 1 1 2 3 2 3 3 2 2 1 3 1 3 2 1 1 1 3 2 2 3 2 3
## [176] 3 2 2 2 1 1 2 1 2 1 1 1 2 1 3 2 3 2 2 1 2 3 2 1 1 1 2 3 2 2 2 3 2 3 2
## [211] 2 3 1 1 2 2 2 2 2 1 3 2 3 1 1 2 3 1 2 3 1 3 1 1 3 3 3 1 1 2 3 1 1 3 3
## [246] 1 1 2 2 2 1 3 3 2 1 3 2 3 2 1 1 2 3 1 3 1 3 2 3 1 3 2 1 1 1 1 3 3 1 2
## [281] 1 1 1 3 3 3 2 1 2 2 1 1 1 1 1 1 2 2 1 1 2 1 3 1 1 1 1 1 1 3 3 2 1 3 2
## [316] 1 3 1 3
## 
## Within cluster sum of squares by cluster:
## [1] 139.99603  92.20125 148.38889
##  (between_SS / total_SS =  59.5 %)
## 
## Available components:
## 
## [1] "cluster"      "centers"      "totss"        "withinss"    
## [5] "tot.withinss" "betweenss"    "size"         "iter"        
## [9] "ifault"
\end{verbatim}

\includegraphics{Clustering_Analysis_in_R_files/figure-latex/unnamed-chunk-6-1.pdf}
\includegraphics{Clustering_Analysis_in_R_files/figure-latex/unnamed-chunk-6-2.pdf}
\includegraphics{Clustering_Analysis_in_R_files/figure-latex/unnamed-chunk-6-3.pdf}
\includegraphics{Clustering_Analysis_in_R_files/figure-latex/unnamed-chunk-6-4.pdf}


\end{document}
